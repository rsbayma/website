\documentclass[12pt, a4paper]{article}
\setlength{\oddsidemargin}{0.5cm}
\setlength{\evensidemargin}{0.5cm}
\setlength{\topmargin}{-1.6cm}
\setlength{\leftmargin}{0.3cm}
\setlength{\rightmargin}{0.3cm}
\setlength{\textheight}{24.00cm} 
\setlength{\textwidth}{15.00cm}
\parindent 20pt
\parskip 5pt
\pagestyle{plain}

\usepackage{graphicx}
\usepackage{epstopdf}
\usepackage{amssymb}
\usepackage{mathtools}
\usepackage{bm}
\usepackage{booktabs}
\usepackage[utf8]{inputenc}
\usepackage{fontenc}
\usepackage[brazilian]{babel}
\usepackage{float}
\usepackage{subcaption}
\usepackage{colortbl} 
\usepackage[dvipsnames]{xcolor} 
\usepackage{enumerate}
\usepackage{cancel}

\title{Solução de equação de diferenças}

\date{}

\allowdisplaybreaks

\newcommand{\namelistlabel}[1]{\mbox{#1}\hfil}
\newenvironment{namelist}[1]{%1
	\begin{list}{}
		{
			\let\makelabel\namelistlabel
			\settowidth{\labelwidth}{#1}
			\setlength{\leftmargin}{1.1\labelwidth}
		}
	}{%1
\end{list}}

\begin{document}
	\maketitle

\section{ }

\newcommand{\Z}[1]{\mathcal{Z}\left\{#1\right\}}
\newcommand{\iZ}[1]{\mathcal{Z}^{-1}\left\{#1\right\}}
\begin{align}
	2y[k+1]-y[k]&=3x[k]\\
\end{align}

Use:
\begin{align}
	x[k] &= u[k]\\
	y[0] &= 0\\
	\Z{a^ku[k]} &= \frac{z}{z-a}
\end{align}


Solução:
\begin{align}
	\Z{2y[k+1]-y[k]}&=\Z{3x[k]}\\
	2\Z{y[k+1]}-\Z{y[k]}&=3\Z{x[k]}\label{eq:eq transformada}
\end{align}

\begin{align}
	\Z{y[k]} &= Y(z)\\
	\Z{x[k]} &= X(z)\\
	\Z{y[k+1]} &= z\,Y(z)
\end{align}

Assim, de \eqref{eq:eq transformada}
\begin{align}
	2zY(z)-Y(z)&=3X(z)\\
	(2z-1)Y(z) &= 3X(z)\\
	\frac{Y(z)}{X(z)} &= \frac{3}{2z-1}
\end{align}

\pagebreak

Como $x[k]=u[k]$, então $X(z)=\Z{u[k)}=z/(z-1)$. Então:
\begin{align}
Y(z) &= \frac{3X(z)}{2z-1} = \frac{3z}{(z-1)(2z-1)}\\
&= \frac{3z}{2(z-1)(z-0.5)}= \frac{1.5z}{(z-1)(z-0.5)}\\
\Rightarrow \frac{Y(z)}{z} &= \frac{1.5}{(z-1)(z-0.5)}
\end{align}

\begin{align}
	\frac{Y(z)}{z} &= R(z)= \frac{A_1}{z-1}+\frac{A_2}{z-0.5}\\
	A_1 &= \left.R(z)(z-1)\right|_{z=1}\\
	A_2 &= \left.R(z)(z-0.5)\right|_{z=0.5}
\end{align}

\begin{align}
	A_1 &= \left.\frac{1.5z}{\cancel{(z-1)}(z-0.5)}\cancel{(z-1)}\right|_{z=1}\\
	A_1 &=\frac{1.5\cdot 1}{1-0.5} = 3\\
	A_2 &= \left.\frac{1.5z}{\cancel{(z-0.5)}(z-1)}\cancel{(z-0.5)}\right|_{z=0.5}\\
	A_2 &=\frac{1.5\cdot 0.5}{0.5-1} = -1.5
\end{align}
\pagebreak


\begin{align}
	\frac{Y(z)}{z} &= \frac{3}{z-1}-\frac{1.5}{z-0.5}\\
	Y(z) &= \frac{3z}{z-1}-\frac{1.5z}{z-0.5}
\end{align}

\begin{align}
	y[k]&=\iZ{Y(z)}\\
	y[k]&=\iZ{\frac{3z}{z-1}-\frac{1.5z}{z-0.5}}\\
	&= 3\iZ{\frac{z}{z-1}}-1.5\iZ{\frac{z}{z-0.5}}\\
	&= 3u[k]-1.5 (0.5)^ku[k]
\end{align}

\pagebreak
\section{ }

\begin{align}
	10y[k+2]&=9y[k]-y[k+1]+5x[k+1]+x[k]\\
\end{align}

Use condições iniciais nulas e:
\begin{align}
	x[k] &= \left(\frac{1}{2}\right)^k \,u[k]\\
	\Z{a^ku[k]} &= \frac{z}{z-a}
\end{align}

\begin{align}
	\Z{10y[k+2]}&=\Z{9y[k]-y[k+1]+5x[k+1]+x[k]}\\
	10\Z{y[k+2]}&=9\Z{y[k]}-\Z{y[k+1]}+5\Z{x[k+1]}+\Z{x[k]}\label{eq:eq transformada2}
\end{align}

No entanto:
\begin{align}
	\Z{y[k]} &= Y(z)\\
	\Z{x[k]} &= X(z) = \frac{z}{z-0.5}\\
	\Z{y[k+1]} &= z\,Y(z)\\
	\Z{y[k+2]} &= z^2\,Y(z)\\
	\Z{x[k+1]} &= z\,X(z)
\end{align}

Assim, de \eqref{eq:eq transformada2}
\begin{align}
	10z^2Y(z)&=9Y(z)-zY(z)+5zX(z)+X(z)\\
	\frac{Y(z)}{X(z)} &= \frac{5z+1}{10z^2+z-9}
\end{align}
\pagebreak


Agora:
\begin{align}
	Y(z) &= \frac{5z+1}{10z^2+z-9}X(z)\\
	&= \frac{5z+1}{10z^2+z-9}\,\frac{z}{z-0.5}\\
	\Rightarrow \frac{Y(z)}{z} &= \frac{5z+1}{(10z^2+z-9)(z-0.5)} = 
	\frac{5z+1}{10(z+1)(z-0.5)(z-0.9)} 
\end{align}

Assim:
\begin{align}
	\frac{Y(z)}{z} &= R(z)= \frac{A_1}{z+1}+\frac{A_2}{z-0.5}+\frac{A_3}{z-0.9}\\
	A_1 &= \left.R(z)(z+1)\right|_{z=-1}\\
	A_2 &= \left.R(z)(z-0.5)\right|_{z=0.5}\\
	A_3 &= \left.R(z)(z-0.9)\right|_{z=0.9}	
\end{align}

\begin{align}
	A_1 &= \left.\frac{5z+1}{10\cancel{(z+1)}(z-0.5)(z-0.9)}\cancel{(z+1)}\right|_{z=-1}\\
	A_1 &=\frac{-5+1}{10(-1-0.5)(-1-0.9)} \approx -0.1403\\
	A_2 &= \left.\frac{5z+1}{(z+1)\cancel{(z-0.5)}(z-0.9)}\cancel{(z-0.5)}\right|_{z=0.5}\\
	A_2 &=\frac{5\cdot 0.5+1}{(0.5+1)(0.5-0.9)} \approx -0.5833\\
	A_3 &= \left.\frac{5z+1}{(z+1)(z-0.5)\cancel{(z-0.9)}}\cancel{(z-0.9)}\right|_{z=0.9}\\
	A_3 &=\frac{5\cdot 0.9+1}{(0.9+1)(0.9-0.5)} \approx 0.7237
\end{align}
\pagebreak

\begin{align}
	\frac{Y(z)}{z} &= \frac{-0.1403}{z+1}-\frac{0.5833}{z-0.5}+\frac{0.7237}{z-0.9}\\
	Y(z) &= \frac{-0.1403z}{z+1}-\frac{0.5833z}{z-0.5}+\frac{0.7237z}{z-0.9}
\end{align}

\begin{align}
	y[k]&=\iZ{Y(z)}\\
	y[k]&=\iZ{\frac{-0.1403z}{z+1}-\frac{0.5833z}{z-0.5}+\frac{0.7237z}{z-0.9}}\\
	&= -0.1403\iZ{\frac{z}{z+1}}-0.5833\iZ{\frac{z}{z-0.5}}\nonumber\\&\,+0.7237\iZ{\frac{z}{z-0.9}}\\
	&= \left(0.7237(0.9)^k--0.5833(0.5)^k-0.1403(-1)^k\right)u[k]
\end{align}

\end{document}
