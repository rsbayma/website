\documentclass[12pt, a4paper]{article}
\setlength{\oddsidemargin}{0.5cm}
\setlength{\evensidemargin}{0.5cm}
\setlength{\topmargin}{-1.6cm}
\setlength{\leftmargin}{0.3cm}
\setlength{\rightmargin}{0.3cm}
\setlength{\textheight}{24.00cm} 
\setlength{\textwidth}{15.00cm}
\parindent 20pt
\parskip 5pt
\pagestyle{plain}

\usepackage{graphicx}
\usepackage{epstopdf}
\usepackage{amssymb}
\usepackage{mathtools}
\usepackage{bm}
\usepackage{booktabs}
\usepackage[utf8]{inputenc}
\usepackage{fontenc}
\usepackage[brazilian]{babel}
\usepackage{float}
\usepackage{subcaption}
\usepackage{colortbl} 
\usepackage{xcolor} 
\usepackage{enumerate}

\title{Guia de análise em tempo discreto}
\date{\today}

\allowdisplaybreaks

\newcommand{\namelistlabel}[1]{\mbox{#1}\hfil}
\newenvironment{namelist}[1]{%1
\begin{list}{}
    {
        \let\makelabel\namelistlabel
        \settowidth{\labelwidth}{#1}
        \setlength{\leftmargin}{1.1\labelwidth}
    }
  }{%1
\end{list}}

\begin{document}
\maketitle

\section{Transformada Z}

Transformadas fundamentais:
\begin{align}
	x[k] &=\delta[k] \leftrightarrow X(z) = 1\\
	x[k] &= a^ku[k] \leftrightarrow X(z) = \frac{z}{z-a}
\end{align}

Obs: quando $a$ for complexo, use módulo-fase e procure pelos pares conjugados.

Propriedades fundamentais da transformada Z

\newcommand{\Z}[1]{\mathcal{Z}\left\{#1\right\}}
Linearidade:
\begin{align}
	\Z{a_1x_1[k]+a_2x_2[k]} &= a_1X_1(z) + a_2X_2(z)
\end{align}

Avanço temporal e condição inicial:
\begin{align}
	\Z{x[k+1]} &= zX(z)-zx[0] 
\end{align}

Convolução. Se:
\begin{align}
 	x[k]*h[k] &= \sum_{n=-\infty}^{+\infty} h[n]x[k-n]
\end{align}
então:
\begin{align}
	\Z{x[k]*h[k]} &= X(z)H(z)
\end{align}

\section{Escolha do período de amostragem}

Sistema com polos reais: use período de amostragem 10x menor que a \textbf{menor constante de tempo}

Sistema com polos complexos: use frequência de amostragem 10x maior que a \textbf{maior frequência natural}.

\textbf{Outra opção: } use período de amostragem menor que 10x o tempo de acomodação mais rápido.

\section{Fundamentos de análise discreta}

Relação de mapeamento:
\begin{align}
	z = e^{sT}
\end{align}

Eixo imaginário: $s=j\omega$ $\rightarrow$ Circunferência de raio 1: $|z|=1$

Semi-plano esquerdo: $\Re\{s\}<0$ $\rightarrow$ Interior do círculo unitário: $|z|<1$

Infinito esquerdo: $\Re\{s\}=-\infty$ $\rightarrow$ Origem: $z=0$

Pólos:

\begin{itemize}
\item Real puro positivo, $|z|<1$. Exponencial pura, estável (decrescente)

\item Real puro positivo, $|z|>1$. Exponencial pura, não-estável (crescente)

\item Real puro negativo, $|z|<1$. Exponencial com sinal alternante, estável (decrescente)

\item Real puro positivo, $|z|>1$. Exponencial com sinal alternante, não-estável (crescente)

\item $z=1$. Degrau

\item $z=-1$. Impulsos alternantes

\item Qualquer imaginário $|z|=1$. Senóide pura. Frequência = fase do polo.

\item Complexo $|z|<1$. Senóide amortecida. 
\end{itemize}

\section{Teorma do valor final}

Se o sinal no domínio do tempo converge para um valor constante (degrau), então o valor final é dado por:
\begin{align}
	y[\infty] = \lim\limits_{z\rightarrow 1} \left(\frac{z-1}{z}\right) Y(z)
\end{align}

Se o sinal tiver uma rampa, ao invés de um degrau, na sua transformada, aumente a potência do fator $\frac{z-1}{z}$ até a rampa sumir. 

\section{Equivalentes discretos}

\subsection{Tustin, bilinear ou trapezoidal}

\begin{align}
	s = \frac{2}{T}\,\frac{z-1}{z+1}
\end{align}

Em alguns casos também é bom saber a relação inversa:
\begin{align}
	z &= \frac{T}{2} \,\frac{s+1}{s-1}
\end{align}

\subsection{Mapeamento}

Proponha uma função de transferência $\hat{G}(z)$ no domínio Z que tenha o mesmo número de polos e zeros explícitos da função contínua $G(s)$. Deixe o ganho na função em Z implícito, isto é, denotado por uma variável $K$, por exemplo.

Para cada polo e zero explícito na função contínua, obtenha um polo ou zero explícito na função discreta usando a relação:
\begin{align}
	z = e^{sT}
\end{align}

Se achar necessário, introduza zeros em $z=-1$ na função discreta. Lembre-se que só se pode introduzir zeros até o limite máximo em que o número de pólos se iguala ao número de zeros da nova função. 

Equalize os ganhos DC das duas funções e resolva a equação para obter o ganho $K$ da função discreta:
\begin{align}
	G(0) &= \hat{G}(1)
\end{align}

\subsection{Equivalente por holder}

\begin{align}
	\hat{G}(z) &= \frac{z-1}{z} \mathcal{Z}\left\{\frac{G(s)}{s}\right\}
\end{align}

Para achar a transformada $\mathcal{Z}$ de uma função de $s$:
\begin{enumerate}
	\item Expanda $G(s)/s$ em frações parciais
	\item Ache a transformada de Laplace inversa
	\item Substitua $t=kT$
	\item Calcule a transformada $\mathcal{Z}$ da função de tempo discreto resultante. 
\end{enumerate}

\section{Projeto por emulação}

Dada a função de transferência contínua da planta, projete um controlador também contínuo. Escolha um período de amostragem apropriado e discretize usando um método à sua escolha.

\section{Diagramas de blocos equivalentes}

Dados dois blocos $G(z)$ e $C(z)$ em série, paralelo ou loop, as regras de simplificação em bloco equivalente $M(z)$ seguem as mesmas do domínio contínuo

Paralelo = mesma entrada, saídas ligadas no somador
\begin{align}
	M(z) = G(z) + C(z)
\end{align}

Série = entrada de um é a saída do outro
\begin{align}
	M(z) = G(z)C(z)
\end{align}

Malha, loop: saída de um é realimentada para a própria entrada através do outro e um somador
\begin{align}
	M(z) = \frac{G(z)}{1+C(z)G(z)}
\end{align}

\section{LGR discreto}

Segue as mesmas regras de construção do LGR contínuo. O projeto é feito apenas observando o círculo unitário como referência de estabilidade.

Lembre-se: equação fundamental (supõe controle proporcional)
\begin{align}
	1 + K\, G(z) = 0
\end{align}

Condição de fase:
\begin{align}
	\angle G(z) = \pm 180^o
\end{align}

Condição de módulo
\begin{align}
	|K G(z)| = 1
\end{align}

Pontos de destaque do LGR (revisão)
\begin{itemize}
	\item Pontos de partida: pólos de malha aberta
	\item Pontos de chegada: zeros de malha aberta ou assíntotas
	\item Número de assíntotas: número de pólos menos o número de zeros
	\item Estabilidade crítica: cruzamento com o círculo unitário
	\item Pontos de ramificação (raízes múltiplas)
\end{itemize}

\end{document}

