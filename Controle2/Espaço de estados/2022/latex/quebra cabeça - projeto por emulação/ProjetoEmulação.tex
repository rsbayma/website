\documentclass[12pt, a4paper]{article}

\setlength{\oddsidemargin}{0.5cm}
\setlength{\evensidemargin}{0.5cm}
\setlength{\topmargin}{-1.6cm}
\setlength{\leftmargin}{0.3cm}
\setlength{\rightmargin}{0.3cm}
\setlength{\textheight}{24.00cm} 
\setlength{\textwidth}{15.00cm}
\parindent 20pt
\parskip 5pt
\pagestyle{plain}

\usepackage{graphicx}
\usepackage{epstopdf}
\usepackage{amssymb}
\usepackage{mathtools}
\usepackage{bm}
\usepackage{booktabs}
\usepackage[utf8]{inputenc}
\usepackage{fontenc}
\usepackage[brazilian]{babel}
\usepackage{float}
\usepackage{subcaption}
\usepackage{colortbl} 
\usepackage[dvipsnames]{xcolor} 
\usepackage{enumerate}
\usepackage{cancel}

\title{Projeto de controlador usando ``emulação''}

\date{}

\allowdisplaybreaks

\newcommand{\namelistlabel}[1]{\mbox{#1}\hfil}
\newenvironment{namelist}[1]{%1
	\begin{list}{}
		{
			\let\makelabel\namelistlabel
			\settowidth{\labelwidth}{#1}
			\setlength{\leftmargin}{1.1\labelwidth}
		}
	}{%1
\end{list}}

\begin{document}
	\maketitle

\section{Problema 1}

\newcommand{\Z}[1]{\mathcal{Z}\left\{#1\right\}}
\newcommand{\iZ}[1]{\mathcal{Z}^{-1}\left\{#1\right\}}

Sistema:
\begin{align}
	G(s) &= \frac{10}{2s+3}
\end{align}

Especificações: $t_s\leq 1$, $M_p\leq 12\%$.

\begin{align}
	M_p = \exp\left(\frac{-\pi\xi}{\sqrt{1-\xi^2}}\right) \Leftrightarrow \xi = \frac{-\log M_p}{\sqrt{\pi^2+(\log M_p)^2}}
\end{align}

Para $M_p=0.12$, $\xi\approx 0.56$.

algum texto

\begin{align}
	t_s &= \frac{4}{\xi \omega_n}	\Rightarrow \omega_n = \frac{4}{\xi t_s}\\
	\omega_n &\approx 7.15 \text{ rad/s}
\end{align}

Polos e polinômio desejados:
\begin{align}
	p_{1,2} &= -\xi\omega_n \pm j\omega_n\sqrt{1-\xi^2}\approx -4\pm j5.93\\
	P(s)  &= (s+4-j5.93)(s+4+j5.93) = s^2+8s+51.12
\end{align}

Controlador PI:
\begin{align}
	C(s) &= k_p+\frac{k_i}{s} = \frac{k_ps+k_i}{s}
\end{align}

\begin{align}
	1+C(s)G(s) &= 1 +\frac{k_ps+k_i}{s}\frac{10}{2s+3}\\
	&= \frac{2s^2+3s+10k_ps+10k_i }{s(2s+3)}\\
	&= \frac{2s^2+(3+10k_p)s+10k_i }{s(2s+3)}
\end{align}

Logo, o polinômio de malha fechada é:
\begin{align}
	2s^2+(3+10k_p)s+10k_i &= 2\left(	s^2+\frac{3+10k_p}{2}s+5k_i\right)
\end{align}

\begin{align}
s^2+\frac{3+10k_p}{2}s+5k_i &= s^2+8s+51.12
\end{align}

Logo:
\begin{align}
	\frac{3+10k_p}{2} &= 8 \Rightarrow k_p = 1.3\\
	5k_i &= 51.12 \Rightarrow k_i = 10.224\\
	\Rightarrow C(s) &= 1.3 + \frac{10.224}{s}
\end{align}

Frequência de amostragem: $\omega_s = 2\pi f_s$
\begin{align}
	\omega_s &\geq 10\omega_n\\
	\omega_s &\geq 71.5\\
	f_s &\geq \frac{71.5}{2\pi} \approx 11.38 \text{ Hz}
\end{align}

Podemos escolher então $f_s=16$ Hz. Logo $T=1/f_s=62.5$ milisegundos. 

\pagebreak
Método trapezoidal, bilinear ou de Tustin:
\begin{align}
	s \leftarrow \frac{2}{T}\,\frac{z-1}{z+1}
\end{align}

\begin{align}
	\frac{1}{T} &= 16\\
	s &\leftarrow 32 \frac{z-1}{z+1} \\
	\frac{1}{s}&\leftarrow \frac{z+1}{32(z-1)}
\end{align}

Logo:
\begin{align}
	C(z) &= 1.3 + \frac{10.22}{s}\\
	&= 1.3 + 10.22 \frac{z+1}{32(z-1)}\\
	&= \frac{32\cdot 1.3(z-1)+10.22(z+1)}{32(z-1)}\\
	&= \frac{51.82z-31.38}{32(z-1)}\\
	&= \frac{1.62z-0.98}{z-1} \cdot \frac{z^{-1}}{z^{-1}}\\
	&= \frac{1.62-0.98z^{-1}}{1-z^{-1}}
\end{align}

\pagebreak
Implementação:
\begin{align}
	C(z) &= \frac{U(z)}{E(z)}
\end{align}
onde:
\begin{align}
	U(z) &= \Z{u[k]}\\
	E(z) &= \Z{e[k]}
\end{align}

\begin{align}
	\frac{U(z)}{E(z)} &= \frac{1.62-0.98z^{-1}}{1-z^{-1}}\\
	(1-z^{-1})U(z) &= (1.62-0.98z^{-1})E(z)\\
	U(z) - z^{-1}U(z) &= 1.62E(z) -0.98z^{-1}E(z)\\
	\iZ{U(z) - z^{-1}U(z)} &= \iZ{1.62\,E(z) -0.98z^{-1}E(z)}\\
	u[k] -u[k-1] &= 1.62 e[k] -0.98 e[k-1]\\
	u[k] &= u[k-1]+ 1.62 e[k] -0.98 e[k-1]
\end{align}

\newcommand\tab[1][1cm]{\hspace*{#1}}
Código de implementação:\\
	float uk=0;\\
	float uk1 = 0;\\
	float ek=0;\\
	float ek1=0;\\
	float y;\\
	int k=0;\\
	float r=1;\\
	float T = 0.0625;\\
	float Tm = 1000*T;\\
	while \{\\
		\tab y = readAnalog(A1);\\
		\tab ek = r-y;\\
		\tab uk = uk1 +1.62*ek-0.98*ek1;\\
		\tab analogWrite(uk);\\
		\tab ek1 = ek;\\
		\tab uk1 = uk;\\
		\tab k = k+1;\\
		\tab delay(Tm)	\\
	\}\\

\end{document}
