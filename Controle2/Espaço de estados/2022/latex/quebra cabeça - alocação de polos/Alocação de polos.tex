\documentclass[12pt, a4paper]{article}
\setlength{\oddsidemargin}{0.5cm}
\setlength{\evensidemargin}{0.5cm}
\setlength{\topmargin}{-1.6cm}
\setlength{\leftmargin}{0.3cm}
\setlength{\rightmargin}{0.3cm}
\setlength{\textheight}{24.00cm} 
\setlength{\textwidth}{15.00cm}
\parindent 20pt
\parskip 5pt
\pagestyle{plain}

\usepackage{graphicx}
\usepackage{epstopdf}
\usepackage{amssymb}
\usepackage{mathtools}
\usepackage{bm}
\usepackage{booktabs}
\usepackage[utf8]{inputenc}
\usepackage{fontenc}
\usepackage[brazilian]{babel}
\usepackage{float}
\usepackage{subcaption}
\usepackage{colortbl} 
\usepackage{xcolor} 
\usepackage{enumerate}
\usepackage{cancel}

\title{Projeto por alocação de polos}
\date{\today}

\allowdisplaybreaks

\newcommand{\namelistlabel}[1]{\mbox{#1}\hfil}
\newenvironment{namelist}[1]{%1
\begin{list}{}
    {
        \let\makelabel\namelistlabel
        \settowidth{\labelwidth}{#1}
        \setlength{\leftmargin}{1.1\labelwidth}
    }
  }{%1
\end{list}}

\begin{document}
\maketitle

\section{Problema}

Projete um controlador com ação integral para o sistema 
\begin{align}
	G(z) &= \frac{B(z)}{A(z)} = \frac{(z+1)(z-0.8)}{z(z-0.9)}
\end{align}

De modo que $\xi=0.5$, $\omega_n=2$. Utilize o período de amostragem $T=0.1$.

\section{Solução}

Controlador:
\begin{align}
	C(z) &= \frac{N(z)}{R(z)}
\end{align}
onde $N$ e $R$ são polinômios a se determinar. 

\newcommand{\gr}{\text{grau}}
Suponha que $n=\gr(N)$ e $r = \gr(R)$. Para o controlador ser causal devemos ter $r\geq n$. 

Em malha fechada:
\begin{align}
	s &= -\xi \omega_n \pm j \omega_n\sqrt{1-\xi^2}\\
	&= -1\pm j 2\sqrt{0.75}\approx -1\pm j1.73\\
	z &= \exp(sT) \approx 0.81 \pm j0.15\\
	M(z) &= (z-0.81+j0.15)(z-0.81-j0.15)M'(z) \\
	&=(z^2-1.78z+0.82)M'(z)
\end{align}

Equação de malha:
\begin{align}
	1+C(z)G(z) &= 1+C(z)\,\frac{B(z)}{A(z)}\\
	&= 1+ C(z)\,\frac{(z+1)(z-0.8)}{z(z-0.9)}
\end{align}

Proposta:
\begin{align}
	C(z) &= \frac{N(z)}{(z-0.8)(z-1)R'(z)}
\end{align}

Então:
\begin{align}
	1+C(z)G(z)
	&=1+\frac{N(z)}{\cancel{(z-0.8)}(z-1)R'(z)}\,\frac{(z+1)\cancel{(z-0.8)}}{z(z-0.9)}\\ 
	&= \frac{z(z-1)(z-0.9)R'(z)+(z+1)S(z)}{z(z-0.9)(z-1)R'(z)}
\end{align}

Polinômio de malha:
\begin{align}
z(z-1)(z-0.9)R'(z)+(z+1)N(z) &= M(z)\\
z(z-1)(z-0.9)R'(z)+(z+1)N(z) &= (z^2-1.78z+0.82)M'(z)
\end{align}

Supondo $\gr(R') = r'\geq 0$ e $\gr(M')=m'\geq 0$:
\begin{align}
	 \underbrace{z(z-1)(z-0.9)R'(z)}_{3+r'}+\underbrace{(z+1)N(z)}_{1+n} &= \underbrace{(z^2-1.78z+0.82)M'(z)}_{2+m'}
\end{align}

Para o controlador ser causal, devemos ter $r'+2 \geq n$. Na melhor hipótese $n = r'+2$. Então:
\begin{align}
	r'+3 &= m'+2 \\
	r' &= m'-1 \geq 0
\end{align}
Logo $m'\geq 1$. A situação mais simples (menores graus possíveis) é  $m'=1$ e $r'=0$. 

$m'=1 \Rightarrow$  mais um pólo em malha fechada. Escolhe-se $z=0$. Assim:
\begin{align}
	M'(z) &= z\\
	M(z)&= (z^2-1.78z+0.82)z
\end{align}

$r'=0\Rightarrow$ $n=r'+2=2$. Então:
\begin{align}
	R'(z) &= k_1\\
	N(z) &= k_2z^2+k_3z+k_4
\end{align}

\pagebreak
Voltando à equação de malha:
\begin{align}
	z(z-1)(z-0.9)R'(z)+(z+1)N(z) &= (z^2-1.78z+0.82)M'(z)\\
	z(z-1)(z-0.9)k_1+(z+1)(k_2z^2+k_3z+k_4) &= (z^2-1.78z+0.82)z
\end{align}

Então:
\begin{align}
	k_1+k_2 &=1\\
	-1.9k1 + k_2 + k_3 &= -1.78\\
	0.9k_1 + k_3 + k_4 &= 0.82 \\
	k_4&= 0
\end{align}

Resolvendo obtém-se: $k_1=0.95$, $k_2=0.05$, $k_3=-0.03$, $k_4=0$. Logo:
\begin{align}
	C(z) &= \frac{0.05z^2-0.03z}{0.95(z-1)(z-0.8)}\\
	&= \frac{0.055z^2-0.034z}{(z-1)(z-0.8)}. 
\end{align}

Implementação:
\begin{align}
	C(z) &= \frac{0.055z^2-0.034z}{(z-1)(z-0.8)}\\
	&= \frac{0.055-0.034z^{-1}}{1-1.8z^{-1}+0.8z^{-2}} \\
	\Rightarrow u[k] &= 1.8u[k-1]-0.8u[k-2]+0.55e[k]-0.034e[k-1]
\end{align}



\end{document}

