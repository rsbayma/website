\documentclass[12pt, a4paper]{article}
\setlength{\oddsidemargin}{0.5cm}
\setlength{\evensidemargin}{0.5cm}
\setlength{\topmargin}{-1.6cm}
\setlength{\leftmargin}{0.3cm}
\setlength{\rightmargin}{0.3cm}
\setlength{\textheight}{24.00cm} 
\setlength{\textwidth}{15.00cm}
\parindent 20pt
\parskip 5pt
\pagestyle{plain}

\usepackage{graphicx}
\usepackage{epstopdf}
\usepackage{amssymb}
\usepackage{mathtools}
\usepackage{bm}
\usepackage{booktabs}
\usepackage[utf8]{inputenc}
\usepackage{fontenc}
\usepackage[brazilian]{babel}
\usepackage{float}
\usepackage{subcaption}
\usepackage{colortbl} 
\usepackage{xcolor} 
\usepackage{enumerate}

\title{}
\date{\today}

\allowdisplaybreaks

\newcommand{\namelistlabel}[1]{\mbox{#1}\hfil}
\newenvironment{namelist}[1]{%1
\begin{list}{}
    {
        \let\makelabel\namelistlabel
        \settowidth{\labelwidth}{#1}
        \setlength{\leftmargin}{1.1\labelwidth}
    }
  }{%1
\end{list}}

\begin{document}
\maketitle

Para os problemas a seguir, considere as seguintes funções de transferência 
\begin{align}
	H_1(s) &= \frac{K}{Ts+1}\\
	H_2(s) &= \frac{K}{as+b}\\
	H_3(s) &= \frac{K}{as-b}\\	
	H_4(s) &= \frac{as+b}{cs+d}\\
	H_5(s) &= \frac{K}{(s+a)(s+b)}\\
	H_6(s) &= \frac{K(s+c)}{(s+a)(s+b)}\\
	H_7(s) &= \frac{K(s+d)}{(s+a)(s+b)(s+c)}
\end{align}

Os parâmetros em cada uma podem ser números reais arbitrários.

1 - Determine a constante de tempo dominante de cada uma dos sistemas a seguir
\begin{align}
	H_1(s) &= \frac{1}{2s+1}\\
	H_2(s) &= \frac{10}{s+2}\\
	H_3(s) &= \frac{7}{5s+3}\\
	H_4(s) &= \frac{s+1}{s+2}\\
	H_5(s) &= \frac{1}{(s+3)(3s+1)}\\
	H_6(s) &= \frac{5s+6}{s^2+4s+3}\\
	H_7(s) &= \frac{100}{(10s+5)(6s+7)(20s+53)}
\end{align}

2 - Especifique um período de amostragem apropriado para os sistemas em malha aberta descrito pelas funções.

3 - Especifique um período de amostragem apropriado para os sistemas formados pela realimentação unitárias das funções.

4 - Discretize as equações diferenciais a seguir para um período de amostragem arbitrário $T$, e usando diferenças em avanço e em atraso. 
\begin{align}
	R y(t) + L \dot{y}(t) &= x(t)\\
	m \ddot{y}(t) + k y(t) &= x(t)\\
	\ddot{y}(t) + a y(t) &= bx(t) + c\dot{x}(t)
\end{align}

5 - Discretize as funções de transferência a seguir usando o método trapezoidal.
\begin{align}
	G_1(s) &= \frac{2}{4s+1} & T&=0.1\\
	G_2(s) &= \frac{8}{9s+3} & T&=1\\
	G_3(s) &= \frac{K(s+a)}{(s+b)} & T&\\
	G_4(s) &= \frac{10}{(s+1)(s+2)} & T&=0.01\\	
	G_5(s) &= \frac{4}{s^2+2s+4} & T&=
\end{align}

6 - Faça uma análise crítica dos resultados obtidos no problema 5 usando simulações no computador.

7 - Repita o problema 5 para o método de mapeamento de pólos e zeros. 

8 - Repita o problema 5 usando o método de equivalente ZOH. 

\end{document}

